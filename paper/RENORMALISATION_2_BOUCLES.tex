\documentclass[11pt]{article}
\usepackage{amsmath, amssymb}
\usepackage{geometry}
\setlength{\parskip}{1em}
\geometry{margin=2.5cm}

\title{\textbf{Renormalisation à deux boucles — esquisse compressive}}
\author{Antoine Sekhi}
\date{11 mai 2025}

\begin{document}

\maketitle

\section*{Objectif}

Esquisser la structure des divergences potentielles à deux boucles dans le cadre compressif défini par :
\[ S = \int d^4x \, \sqrt{-g} \left[ \frac{1}{16\pi G} R + \frac{1}{2} \frac{1}{\Phi} g^{\mu\nu} \partial_\mu \Phi \partial_\nu \Phi - V(\Phi) \right] \]
avec :
\[ V(\Phi) = \Phi \ln \Phi + \alpha \Phi^2 \quad \text{et} \quad C(\Phi) = \frac{1}{\Phi} \]

\section{Structure des diagrammes à deux boucles}

Deux types principaux de contributions apparaissent à deux boucles :

\begin{itemize}
  \item \textbf{Diagramme sunset} (auto-énergie à deux propagateurs internes)
  \item \textbf{Diagrammes imbriqués} (bulle dans bulle, corrections de vertex)
\end{itemize}

Le diagramme sunset pour $\Phi$ (interaction $\lambda_4 \Phi^4$ issue du développement de $V(\Phi)$) donne une contribution typique :
\[ \Sigma^{(2)}(p^2) \sim \frac{\lambda_4^2}{(16\pi^2)^2} \left( \Lambda^2 \ln \frac{\Lambda^2}{m^2} + \cdots \right) \]

\section{Compression critique et contrôle des divergences}

Deux mécanismes permettent ici un contrôle compressif :
\begin{itemize}
  \item La structure logarithmique de $V(\Phi)$ implique que les contre-termes au-delà de $\lambda_4$ ne peuvent pas croître librement : ils sont \textit{retenus} par la compression asymptotique.
  \item Le couplage $C(\Phi) = 1/\Phi$ implique une suppression dynamique des hautes fluctuations (\textit{bruit auto-régulé}).
\end{itemize}

\section{Hypothèse compressive}

Nous postulons que :
\begin{quote}
Toute divergence à deux boucles dans ce cadre est soit absorbable par une redéfinition compressive du potentiel $V(\Phi)$, soit supprimée par saturation naturelle des degrés de liberté dans $\Phi$.
\end{quote}

\textit{Autrement dit} : l’espace des contre-termes est compressivement borné. Il n’existe pas de liberté infinie de réécriture à deux boucles dans ce cadre.

\section*{Conclusion provisoire}

\textbf{Résultat compressif v0.1} : Aucune divergence à deux boucles n’apparaît comme non contrôlable dans ce cadre à ce stade. Le potentiel logarithmique, la suppression dynamique, et l’absence de vertex gravitationnel couplé suffisent pour assurer une renormalisabilité compressive partielle.

Un calcul symbolique complet (e.g. via \texttt{sympy} ou \texttt{feyncalc}) est envisagé pour v0.2.

\end{document}
